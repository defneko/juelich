Mechanical heart valves are crucial devices that save thousands of lives each day. Despite their seemingly simple structures, the research and engineering that goes into the design and manufacturing of these devices is extensive. Although these devices are crafted to be nearly flawless, after years of use in a patient's vascular system, imperfections, and surface defects can arise. Sometimes, after many decades of use, the valves need to be replaced. Some of the main factors limiting the valve's performance and service life are the roughening of the surface and the generation and propagation of microcracks. 
\\\\A problem called "cavitation erosion" is identified in some research.\cite*{zhang1995} \cite*{wu2001} This is a situation where pressure differences cause the explosion of oxygen bubbles. This not only reduces the surface quality of these valves but also leads to the initiation of microcracks. In another study conducted in 1998 \cite*{barmada1998} (where 17 valves are examined, most of them being demonstration valves; they only have a few valves that have been used, with the longest usage being 4 years), they correlate the deterioration of valve surface quality with the risk of hemolysis. For instance, they look at the roughness of the valve's thin edge, known as the "knife side." They find that valves removed from patients with complaints of hemolysis have a very rough surface in this area.
\\\\In another study from 2001, \cite*{wu2001} a significant relationship between "cavitation damage" and the lifespan of the valves is emphasized again. Local stresses, prolonged cyclic loading over the years, and cavitation damage can all reduce the valve's lifespan in various ways. One of these is triggering hemolysis due to a decrease in surface quality or the formation of thrombi which usually begins from the hinges, as described in a paper from 2012 \cite*{bonou2012}, and more rarely, the valve's fragmentation and detachment as described in a paper from 2019.\cite*{vansteenbergen2019}
\\\\Consequently, the structural integrity of the valve is of utmost importance. Even valves that appear intact to the naked eye have numerous surface defects. These surface defects lead to various issues, including hemolysis, thrombus formation, and even the fragmentation of the valve.
